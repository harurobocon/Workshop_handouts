\chapter{マネジメント}

この章では,技術的なこと以外で考えて欲しいことについて述べます.

\section{役割分担の視点}

NHKロボコンに出場するようなロボットを,大会までに1人で作り上げることは困難です.
通常はチームで1台もしくは数台のロボットを作り上げます.
従って,複数人で協力して1つのものを作るという作業が必要になるわけですが,ロボットの構成や設計は,時にチームの力を十分に発揮できるかを左右します.

複数人で開発を行うにあたって最も重要なことのうちの1つは,チームメンバー間で「どのようなことを目指して何を作るか」について共通認識を持つということでしょう.
これができて初めて,各チームメンバーが目標に向けてアイデアを出し合うことが可能になります.
共通認識がなければ,有意義な議論をすることは難しいでしょう.
そうはいうものの,共通認識を持つことは容易ではありません.少しでもその状態に近づくために,密なコミュニケーションを取ることが大切です.

その上で,構成を考える段階でロボットの各構成要素がお互いにどのように影響し合うのかということを想定し,お互いの設計部分の仕様をしっかりとすり合わせることが大切です.
構成段階で要素間の関係を意識することによって,バランスの悪いロボットになることも避けられます.例えば,モータに流す必要のある電流と比べてオーバースペックなドライバを作ってしまったり,0.5秒で装填できる装填機構を作ったが実際には5秒かかる移動の最中に装填を行えばよく5秒以内に完了すれば競技達成タイムに影響がなくそこまで高性能にする必要がなかった,といったことを防げたりします.

以上では分担して作業する際にどのように進めると良いかについて議論しました.一方で,分担作業をしやすくするために意識できることとして,構成要素(モジュール)同士の独立性を高めることがあります(これ自体はソフトウェアの分野での言い回しですが,メカや回路についても適用可能な部分はあると思います).
より具体的にいうと,ロボットを構成する要素を機能などの観点で分解し(モジュール化),それぞれ互いに依存する度合いを小さくするということです.依存度合いを減らすとは,一方の仕様合わせて他方の仕様を変える必要性が小さいということなどを指します.
これを実践することによって,お互いの仕様変更の影響を受けにくくなり,作業のやり直しなどがなくなって効率的になることが期待されます.
あとから仕様変更したい際やトラブルが起きた時に他所に影響が及びにくいという利点もあります.
一方で,独立性を高めるあまり性能が低下したり(例えば重量が重くなってしまうとか),コミュニケーションを取る必要が一見なくなり各自がもくもくと作業したのち,後から大きな齟齬が発覚する,などといった副作用が起きるリスクがあります.
1年や半年といった(ロボット開発としては)短い期間で,1台限りのロボットを作る上では,保守性と競技での性能の面でバランスを取る必要があり,一概にどちらが良いと言えないので難しい問題です.難しい問題ではありますが,以上のような観点があるということを意識して取り組むことは重要でしょう.


\section{スケジュールの視点}

ロボコンは,(少なくとも大会成績という観点では)大会当日にロボットが実際に動いて初めて評価される競技です.どんなに素晴らしい目標や構想の上に作られたロボットであっても,動かなければ努力が十分に報われないことになってしまいます.
ロボコンはアイデア対決ではありますがアイデアコンテストではなく,ロボットコンテストなのです.
この点が,ロボコンという競技の厳しくも面白い特徴であります.

大会会場に行くと練習フィールドとは違う様々なトラブルが起きます.このようなことに対応した上で大会当日に確実に動作させるために,ロボットの完成度を高めることが必要です.完成度の高さにはもちろん,洗練された無駄のない動きにより競技を高速にこなすということもありますが,多少のズレなどがあっても競技に支障が出ないように作られていることや,トラブルや例外があったとしてもどうにか対応できるようなことが含まれます.

完成度を高めるためには,「とりあえず完成した」「とりあえず競技課題を一通りこなせた」という状態からさらに改良を積み重ねていくことが必要です.
多少のトラブルがあっても対応できるようにするには起きうるトラブルについて知っていなければなりません.
テストランをとにかくたくさん繰り返せばトラブルを全て洗い出せるとは言いませんが,一定程度の動作試験が必要なことは間違い無いでしょう.
実際に作ってみると思っても見なかった困難が発生するということもあるでしょう.こういうものに対応するのにも時間がかかります.
従って,例え実現できれば間違いなく優勝できるであろうアイデアでも,大会ギリギリになんとか完成できるかなというぐらいの製作難易度のアイデアを採用して突き進むことは必ずしも良い結果をもたらさない,ということです.なぜなら,改良の時間が取れないからです.

ここで,「大会ギリギリになんとか完成できるかな」というスケジュールに関する判断が出てきました.完成度の高いロボットを作る上では,スケジュールの判断を適切に行うことが重要になってきます.この判断は,例えば次の事項に基づいて行われるでしょう.

\begin{itemize}
    \item 自分の過去の経験
    \item 他の人が取り組んだプロジェクトの履歴
    \item プロトタイプを作ってみた結果
\end{itemize}

まず,本資料が対象としている新人大会の出場者の皆さんの多くには,「自分の過去の経験」が無いかもしれません.これは仕方のないことです.

続いて,他の人が取り組んだプロジェクトの履歴です.去年の先輩がどういうスケジュール間で進んだのか,ということが参考にできます.先輩がつけていた作業日誌などを見れば,これがわかるでしょう.初めに立てていたスケジュールに対して遅れや想定外が生じたところは,特に参考になるでしょう.自分のプロジェクトにおいて,トラブルの起きるようなことを避けるとか,余裕を持ったスケジュールを立てるなどの対策が取れるからです.

以上2つの情報もある程度参考になりますが,あくまでも過去の別のロボコンに関することです.NHKロボコンでは毎年ルールが変わりますし,新人大会でもそれに合わせて毎年ルールを新しく作成しています.従って,新しいことをしたりする必要がどうしても生じてくるわけです.そのような時は「うまくいきそうか」,「間に合いそうか」を知るためのプロトタイプを「できるだけ早く」作ることが重要かもしれません.議論したり考えたりしても結論を出せない時には,試作によって何を知りたいのかをきちんと明らかにした上で,それを知るために必要最低限の努力で作れるプロトタイプを素早く製作することも重要です.早い段階で試作に基づいてロボットの構成やスケジュールを決定できれば,完成度の向上に大きく貢献するでしょう.


\section{ブレインストーミングの方法}
NHKロボコンにおいて, 機体の構想を決めるためにはアイデアが非常に重要になります. 
しかしながら, 人間というのは思っているよりも偏見や先入観に囚われやすく, それに伴ってアイデアの幅が狭くなりがちです. 特に, 司会役が各課題に沿って順番に議論及び決定していくようないわゆる「会議」ではその場の雰囲気や, 力関係の強い人(上級生や経験者など)の考えに流されてしまい, 斬新な発想やコンセプトはなかなか発生しにくいです. 

そこで今回は, アイデアの幅を広げる手法の1つとして, ブレインストーミングという方法を紹介します. この方法は広く知られている方法ではありますが, 間違ったやり方がなされていることもあります. ブレインストーミングの特性とやり方をよく理解してから取り組むようにしましょう. 

\subsection{ブレインストーミングの手順}
ブレインストーミングは, 以下の3つのプロセスを必要とします. 
\begin{itemize}
    \item アイデアの発散(爆発)
    \item アイデアの整理
    \item アイデアの収束
\end{itemize}
このプロセスのうち, アイデアの発散(爆発)の部分のみをブレインストーミングと呼ぶ場合がありますが, そのような場合であってもアイデアを発散させた後に整理, 収束させなければ意味をなしません. 整理と収束を行うことによって最終的に良いアイデアをまとめることができます. 

ここでは, 3つのプロセスすべてをまとめてブレインストーミングと呼ぶこととします. 
\subsubsection{アイデアの発散}
アイデアの発散プロセスでは, とにかくアイデアを集めることが重要になります. 1つのアイデアを1つの付箋やカードに整理し, その付箋やカードを1人ずつ出していくという方法が良く用いられます. 
この段階では, アイデアを集めることだけに特化し, 現実味があるかなどはこの時には考えません.
以下に挙げる点に気をつけるようにしましょう. 
\begin{itemize}
    \item 量を重視する(質より量)
    \item 他人のアイデアを否定しない(馬鹿げた考え方も歓迎する)
    \item 他のアイデアに乗っかって発展させる(アイデアの組み合わせや改造)
    \item 判断や結論を出そうとしない(結論厳禁)
\end{itemize}

アイデアを考えるフェイズとそれを発表・共有するフェイズに分けて, この2つのフェイズを複数回繰り返す, という過程を経てアイデアを集めていくと, 他のアイデアからインスピレーションを受けやすいのでおすすめです. アイデアを考えるフェイズではできるだけアイデアを生み出す, 発表・共有するフェイズでは他人の発表をよく聞くといったメリハリをつけると良いでしょう. 

アイデアを考えるフェイズではとにかくアイデアの数が大切になります. どんなにくだらないアイデアでもそれが他の人の発想を刺激することがあるので遠慮せずに出すようにしましょう. この段階での「役に立たないのでは」「現実的でないのでは」という考えはアイデアの幅を狭めることにつながるのでそのようなことを考えないようにすると良いです. 

頑張って捻り出そうとしてもどうしてもアイデアが出ないこともあるかと思います. そのような場合の対処の1つとしてSCAMPER法を紹介します. これは以下の単語の頭文字をとったものです. 
\begin{itemize}
    \item Substitute (入れ替える)他のものに置き換えたらどうか, 代わりになるものはないか, など
    \item Combine (組み合わせる)何かと何かを組み合わせたらどうか, など
    \item Adapt (当てはめる) 似ているものはないか, 過去の例に参考になるものはないか, など
    \item Modify (変更する)大きさ, 形を変えたらどうか, など
    \item Put to other uses (ほかの使い道) 想定されている以外の使い方をしたらどうか, など
    \item Eliminate (削減) 一部を削ってみたらどうか, 何が取り除けるか, など
    \item Reverse・Rearrange (逆転・並べ替え) 逆にしたらどうか, 順序を入れ替えたらどうか, など
\end{itemize}
他人のアイデアや, 自分のアイデアをこれらの方法で大きく増やすことが可能になります. 「意味不明なアイデアがでてきてしまうのでは」と思ってはいけません. 何度も言いますがこの段階ではそれで良いのです. それが役に立つか判断するのは後でやることです.  

発表・共有するフェイズでは発表しやすい雰囲気作りや, 議論が盛り上がるよう工夫することが重要です. 
お互いによく知っている人同士だとあまり問題にならないかと思いますが, 
場合によってはブレインストーミングを始める前にミニゲームを行ってお互いの距離を縮めたりすることも有効です. どうすれば良い雰囲気が作れるか考えてそれぞれのチームに適した方法を取るようにするとよいでしょう.  

\subsubsection{アイデアの整理}
アイデアを考えるフェイズと発表・共有するフェイズを何回も繰り返して十分にアイデアが集まったと判断できたら, 次にアイデアを整理する段階に移ります. アイデアを整理する方法はいくつかありますが, マインドマップを用いて整理する方法や親和図法を用いて整理する方法が比較的有名です. 

マインドマップは概念の中心となるキーワードを中心において, そこから関連のあるキーワードやイメージを放射状につなげて広がっていくように整理する方法です. アイデアを付箋やカードに書いておくとやりやすいでしょう. 

親和図法はアイデアをいくつかのグループに分ける方法です. 似たアイデアを同じグループに分け, その類似点からグループに名前をつけます. そして, まとめたグループも似ている部分があればさらに上位となる大グループを作っていく……というふうにして整理していきます. こちらも付箋やカードにアイデアが整理されているとやりやすいでしょう. 

アイデアの整理を行うことで, 全体像が見えてきます. また, 整理したことによって新たにアイデアを思いついたり, 足らない部分が見えてくることもあるでしょう. その場合はもう一度アイデアを発散させる段階に戻っても良いでしょう. アイデアの発散と整理を繰り返すことでより多くのアイデアを集められます. この段階までで出来るだけアイデアを出し切るようにしましょう. 

\subsubsection{アイデアの収束}
アイデアを出し切ったら, これを収束させていくことが必要です. ここでようやく実現性などを考慮する段階に入ります. 

いくつもあるアイデアから, どのアイデアが現実的であるか, 実際にどのアイデアを採用するかを技術的な観点に加えて, 役割分担やスケジュールの視点からも考慮し, 総合的に判断するようにしましょう. 判断がしにくいアイデアについては, 時間などのリソースがある場合は実際にプロトタイプを制作してみてから判断するという方法もあるでしょう. ただし, 最終的には結論を1つに収束させることが必要です. いつまでもコンセプトが定まらないと制作するロボットの全体像が見えず, チームが混乱することも考えられます. どのアイデアを採用するかで延々と時間を消費しないように気をつけましょう. 